\section{Results using RTL-SDR}

Both the Cumulant method and the Cyclostationary method were rewritten in C++
and placed into a gnuRadio block for use with the RTL-SDR receiver.  Without a
method of generating signals, commonly known signal sources had to be used for
experimentation.  Known signals that are within the frequency range of
the receiver included FM Broadcasts and the BPSK modulated RDS data.

Figure~\ref{fig:SignalOfInterest} shows an averaged Frequency Spectrum from the
RTL-SDR tuned to 88.1Mhz with the sample rate set to 2MSpS.  The FM radio
station at 88.5Mhz is very prevalent and the square signals located to either
side of the signal is either the RDS (Radio Description Service) or the
digital HD Radio data.  

\begin{figure}
\centering
\includegraphics[width=\linewidth]{../img/Report_RTL_SDR_FFT_881M_2Msps.png}
\caption{Spectrum of FM Radio Station located at 88.5 Mhz}
\label{fig:SignalOfInterest}
\end{figure}


\subsection{RTL-SDR Cumulant Results}

Time domain classification of unknown signals presents multiple problems
for the implementer.  In simulations, the exact frequency, phase, symbol rate,
and matched filter of the signal is known.  Without this prior knowledge, or the
means of detecting it, blind modulation detection from the time domain becomes a
difficult task.
 
Prior to using the cumulant method to classify the signal, the data was
frequency shifted, filtered, and decimated.  Since no phase lock loop was used,
it is expected that the frequency of the signal of interest will be slightly
different than the tuned frequency.  This will result in a continually rotating
IQ plane.
Earlier it was stated that the Cumulant method was phase invariant however this
only applies to a static phase shift and not a continuous rotation.  Without a phase
lock loop, the IQ plane of FM, BPSK, and QAM will be indistinguishable and
resemble a circle with added noise.
AM, SSB-AM, and higher order QAM signals which would not form a single circle in
the IQ domain may still be able to be identified but signal sources of this type
within the frequency range of the RTL-SDR are difficult to find.
Figure~\ref{fig:digitalIQ}a shows the spectrum of the digital signal to the
left of the FM station.  This signal was tuned, filtered, and decimated to
isolate the digital signal.  The IQ plot of this data, shown in
Figure~\ref{fig:digitalIQ}b, does not provide any visual information about its
content since the SNR is only 10dB and, the symbol rate, and exact
frequency, are unknown.  The Cumulant method fails to classify the signal
correctly and instead classifies it as an AM signal with $C40 = 0.07$ and $C42 = 0.535$.

The same process was used to tune, filter, and decimate the FM signal source. 
Despite the clearer IQ plot, due to the higher SNR and shown in
Figure~\ref{fig:FMPlots}, the FM signal is also miss-classified as AM with $C40 = 0.021$ and $C42 =  0.535$.

\begin{figure*}
\centering
\begin{subfigure}{.49\textwidth}
\centering
\includegraphics[width=\linewidth]{../img/Report_Decimated_Digital_Signal.png}
\caption{ }
\end{subfigure}
\begin{subfigure}{.49\textwidth}
\centering
\includegraphics[width=\linewidth]{../img/Report_IQ_Plot_Digital.png}
\caption{ }
\end{subfigure}
\caption{Digital Signal FFT (a) and IQ domain plot (b)}
\label{fig:digitalIQ}
\end{figure*}

\begin{figure*}
\centering
\begin{subfigure}{.49\textwidth}
\centering
\includegraphics[width=\linewidth]{../img/Report_Decimated_FM_Signal.png}
\caption{ }
\end{subfigure}
\begin{subfigure}{.49\textwidth}
\centering
\includegraphics[width=\linewidth]{../img/Report_IQ_FM_Data.png}
\caption{ }
\end{subfigure}
\caption{FM Signal FFT (a) and IQ domain plot (b)}
\label{fig:FMPlots}
\end{figure*}

\subsection{RTL-SDR Cyclostationary Results}

The cyclostationary approach discussed earlier depended
on a known template for pattern matching the cycle frequency domain profile. 
This CDP is directly linked to the shape of the frequency domain plot of the
signal.  In real world applications, this shape is based off of factors
including the matched filter, and symbol rate for digital signals and the
frequency deviation and input bandwidth for FM and AM sources.
Modulation classification using the previous made templates could not be
expected to do well.  Not only will the unknown variables cause issue but the templates
assumed the signal was the only signal present which is not the case in real the
world.  Filtering out the other signals would be a possibility but it would also
greatly reduce the noise floor compared to the noise floor of the template
causing additional problems.

Rather than use the cyclostationary detection methods for identification of a
particular modulation based on the previous templates, new templates should be
generated using a known signal and the cyclostationary method should be used to
identify similar signals throughout the spectrum.  

To obtain a spectral template of an FM signal with digital RDS data, the signal
was centered at 88.5Mhz, filtered to 600khz and down sampled by a factor of four.
Figure~\ref{fig:decimatedFMandDig} shows the resulting signal. 
Figure~\ref{fig:cycleSpec885} shows the Cycle Spectral Correlation and the cycle
domain profile.

\begin{figure}
\centering
\includegraphics[width=\linewidth]{../img/Report_Decimated_FM_and_Digital.png}
\caption{Decimated FM signal}
\label{fig:decimatedFMandDig}
\begin{comment}
[Sxa Ia] = detectWithRTL_Cyclo(88500000, 4, 600000);
\end{comment}
\end{figure}

\begin{figure*}
\centering
\begin{subfigure}{.49\textwidth}
\centering
\includegraphics[width=\linewidth]{../img/Report_Cycle_Spectral_Corr_88_5}
\begin{comment}
[Sxa Ia] = detectWithRTL_Cyclo(88500000, 4, 600000);
\end{comment} 
  \caption{ }
\end{subfigure}
\begin{subfigure}{.49\textwidth}
  \centering
  \includegraphics[width=\linewidth]{../img/Report_CDP_885.png}
  \begin{comment}
[Sxa Ia] = detectWithRTL_Cyclo(88500000, 4, 600000);
\end{comment}
  \caption{ }
\end{subfigure}
\caption{Cycle Spectral Correlation (a) and Cycle Domain Profile (b)
for the FM signal with digital side bands}
\label{fig:cycleSpec885}
\end{figure*}