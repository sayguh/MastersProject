\chapter{Simulation Results}

Both methods were tested in Octave at -3dB, 0dB, 3dB, 5dB, and 10dB SNR.  As
test data, AM, Single Side Band AM, FM, BPSK, QAM, 16QAM, and 64QAM data was
used.  Each signal was modulated at 2048Hz and sampled at 8192Hz with complex
sampling being used.  This resulted in an over sampling rate of four.  For each
of the digital modulation schemes, a symbol rate equal to the center frequency
of 2048Hz was used.  For the analog modulated signals, white noise was filtered
to a bandwidth of 2048Hz as the input signal.  The FM frequency deviation was
set to 50Hz.

\section{Cumulant Method}

While \cite{swami2000} does not cover FM Modulation, an FM signal in the IQ
domain can be seen has a rotating phasor which given that the Cumulant Method is
phase invariant should result in the classification of M-PSK signal with M
approaching infinity.  
For the AM and SSB AM, data with an SNR of 1000dB was sent through the
classifier to determine the theoretical values of $|C40|$ and $C42$.  For AM,
$|C40|$ and $C42$ were $0.040723$ while SSB data had $|C40| = 0.079237$ and
$C42 = 0$.
The cumulant algorithm works in the time domain and assumes that the signal of
interest has been shifted to baseband.  Prior to running each of the data sets
through the cumulant algorithm, the signal was first frequency shifted to
baseband and then down sampled by a factor of four.  The signal was then sent
into the Cumulant algorithm for classification.  For each of the modulation
types, each correct hit as well as each false positive was noted.  For each
SNR level, one hundred tests were run for each of the modulation types.  The
number of data points was varied and the outcome of each test case is shown in
tables \ref{tab:cumHit100pt} - \ref{tab:cumFalsePositive10000pt}.

\begin{table}
\caption{Cumulant Results, Hits: 100 data Points}
\centering
\begin{tabular}{ l | c | c | c | c | c } \hline
SNR &	 -3 &	 0 &	 3 &	 5 &	 10\\ \hline \hline 
AM &	 12 &	 13 &	 36 &	 41 &	 94 \\ \hline 
SSB &	 32 &	 56 &	 66 &	 71 &	 92 \\ \hline 
FM &	 0 &	 0 &	 0 &	 0 &	 0 \\ \hline 
BPSK &	 0 &	 9 &	 91 &	 100 &	 100 \\ \hline 
QAM &	 0 &	 6 &	 29 &	 85 &	 100 \\ \hline 
QAM16 &	 5 &	 6 &	 8 &	 6 &	 1 \\ \hline 
QAM64 &	 66 &	 76 &	 65 &	 54 &	 29 \\ \hline
\end{tabular}
\label{tab:cumHit100pt}
\end{table}

\begin{table}
\caption{Cumulant Results, False Positives: 100 data Points}
\centering
\begin{tabular}{ l | c | c | c | c | c } \hline
SNR &	 -3 &	 0 &	 3 &	 5 &	 10 \\ \hline \hline
AM &	 1.0 &	 1.6 &	 2.0 &	 1.0 &	 0.4 \\ \hline 
SSB &	 14.9 &	 13.6 &	 9.6 &	 8.9 &	 0.9 \\ \hline 
FM &	 1.9 &	 1.9 &	 1.7 &	 1.9 &	 0.9 \\ \hline 
BPSK &	 0.0 &	 0.7 &	 10.0 &	 14.1 &	 14.3 \\ \hline 
QAM &	 21.1 &	 26.6 &	 5.6 &	 0.1 &	 0.0 \\ \hline 
QAM16 &	 11.6 &	 12.0 &	 15.0 &	 8.7 &	 10.1 \\ \hline 
QAM64 &	 33.1 &	 20.0 &	 14.0 &	 14.3 &	 14.0 \\ \hline
\end{tabular}
\label{tab:cumFalsePositive100pt}
\end{table}


\begin{table}
\caption{Cumulant Results, Hits: 1,000 data Points}
\centering
\begin{tabular}{ l | c | c | c | c | c } \hline
SNR &	 -3 &	 0 &	 3 &	 5 &	 10\\ \hline \hline 
AM &	 25 &	 51 &	 59 &	 80 &	 100 \\ \hline 
 SSB &	 91 &	 88 &	 88 &	 92 &	 100 \\ \hline 
FM &	 0 &	 0 &	 0 &	 0 &	 0 \\ \hline 
BPSK &	 0 &	 0 &	 100 &	 100 &	 100 \\ \hline 
QAM &	 0 &	 0 &	 0 &	 92 &	 100 \\ \hline 
QAM16 &	 0 &	 0 &	 0 &	 0 &	 25 \\ \hline 
QAM64 &	 37 &	 98 &	 100 &	 100 &	 100 \\ \hline
\end{tabular}
\label{tab:cumHit1000pt}
\end{table}

\begin{table}
\caption{Cumulant Results, False Positives: 1,000 data Points}
\centering
\begin{tabular}{ l | c | c | c | c | c } \hline
SNR &	 -3 &	 0 &	 3 &	 5 &	 10 \\ \hline \hline 
AM &	 1.3 &	 1.6 &	 1.7 &	 1.1 &	 0.0 \\ \hline 
SSB &	 28.3 &	 7.4 &	 5.9 &	 2.9 &	 0.0 \\ \hline 
FM &	 0.0 &	 0.1 &	 0.0 &	 0.0 &	 0.0 \\ \hline 
BPSK &	 0.0 &	 0.0 &	 0.0 &	 0.0 &	 14.3 \\ \hline 
QAM &	 11.7 &	 28.6 &	 14.3 &	 14.3 &	 0.0 \\ \hline 
QAM16 &	 15.3 &	 5.7 &	 14.3 &	 1.1 &	 0.0 \\ \hline 
QAM64 &	 21.6 &	 22.7 &	 14.3 &	 14.3 &	 10.7 \\ \hline
\end{tabular}
\label{tab:cumFalsePositive1000pt}
\end{table}

 
\begin{table}
\caption{Cumulant Results, Hits: 10,000 data Points}
\centering
\begin{tabular}{ l | c | c | c | c | c } \hline
SNR &	 -3 &	 0 &	 3 &	 5 &	 10\\ \hline \hline 
AM &	 52 &	 63 &	 93 &	 96 &	 100 \\ \hline 
 SSB &	 92 &	 94 &	 99 &	 100 &	 100 \\ \hline 
FM &	 0 &	 40 &	 99 &	 100 &	 100 \\ \hline 
BPSK &	 0 &	 0 &	 100 &	 100 &	 100 \\ \hline 
QAM &	 0 &	 0 &	 0 &	 100 &	 100 \\ \hline 
QAM16 &	 0 &	 0 &	 0 &	 0 &	 91 \\ \hline 
QAM64 &	 0 &	 100 &	 100 &	 100 &	 100 \\ \hline
\end{tabular}
\label{tab:cumHit10000pt}
\end{table}

\begin{table}
\caption{Cumulant Results, False Positives: 10,000 data Points}
\centering
\begin{tabular}{ l | c | c | c | c | c } \hline
SNR &	 -3 &	 0 &	 3 &	 5 &	 10 \\ \hline \hline
AM &	 1.1 &	 0.9 &	 0.1 &	 0.0 &	 0.0 \\ \hline 
 SSB &	 44.6 &	 5.3 &	 1.0 &	 0.6 &	 0.0 \\ \hline 
FM &	 0.0 &	 0.0 &	 0.0 &	 0.0 &	 0.0 \\ \hline 
BPSK &	 0.0 &	 0.0 &	 0.0 &	 0.0 &	 0.0 \\ \hline 
QAM &	 14.3 &	 14.3 &	 0.0 &	 0.0 &	 0.0 \\ \hline 
QAM16 &	 0.0 &	 2.6 &	 14.3 &	 0.0 &	 0.0 \\ \hline 
QAM64 &	 19.4 &	 34.6 &	 14.4 &	 14.3 &	 1.3 \\ \hline
\end{tabular}
\label{tab:cumFalsePositive10000pt}
\end{table}
 